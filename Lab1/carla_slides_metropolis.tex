\documentclass[t,aspectratio=169]{beamer}

% --- Theme and Color Setup ---
\usetheme{Madrid}

% --- Green palette ---
\definecolor{MyGreen}{RGB}{0,104,55}
\definecolor{MyGreenDark}{RGB}{0,78,41}
\definecolor{MyGreenMid}{RGB}{0,92,48}

\usecolortheme[named=MyGreen]{structure}

% Footer and header colors
\setbeamercolor{palette primary}{bg=MyGreen,fg=white}
\setbeamercolor{palette secondary}{bg=MyGreenDark,fg=white}
\setbeamercolor{palette tertiary}{bg=MyGreenMid,fg=white}
\setbeamercolor{frametitle}{bg=MyGreen,fg=white}
\setbeamercolor{title}{fg=white}
\setbeamerfont{caption}{size=\footnotesize}
\setbeamerfont{author in head/foot}{size=\scriptsize}
\setbeamerfont{title in head/foot}{size=\scriptsize}
\setbeamerfont{date in head/foot}{size=\scriptsize}

% Taller footer bar for extra padding
\setbeamertemplate{footline}{
  \leavevmode%
  \hbox{%
  \begin{beamercolorbox}[wd=.34\paperwidth,ht=3.2ex,dp=1.4ex,center]{author in head/foot}
    \usebeamerfont{author in head/foot}\insertshortauthor
  \end{beamercolorbox}%
  \begin{beamercolorbox}[wd=.33\paperwidth,ht=3.2ex,dp=1.4ex,center]{title in head/foot}
    \usebeamerfont{title in head/foot}\insertshorttitle
  \end{beamercolorbox}%
  \begin{beamercolorbox}[wd=.33\paperwidth,ht=3.2ex,dp=1.4ex,center]{date in head/foot}
    \usebeamerfont{date in head/foot}\insertframenumber/\inserttotalframenumber
  \end{beamercolorbox}}%
  \vskip0pt%
}

% Packages
\usepackage[utf8]{inputenc}
\usepackage{hyperref}
\usepackage{tikz}
\usepackage{booktabs}
\usepackage{tabularx}
\usepackage{colortbl}
\usepackage{xcolor}
\usepackage{listings}
\usepackage{array}

% --- Listings style ---
\lstdefinestyle{code}{
  basicstyle=\ttfamily\small,
  columns=fullflexible,
  frame=single,
  rulecolor=\color{black!20},
  framerule=0.6pt,
  xleftmargin=0.5em,
  xrightmargin=0.5em,
  aboveskip=0.7em,
  belowskip=0.7em,
  showstringspaces=false
}

% --- 1. CONFIGURATION FOR CONTENT SLIDES ---
% Put a white logo in the top right corner of every normal slide title bar
% Replace the filename with the UniBo white logo you download
\addtobeamertemplate{frametitle}{}{%
    \begin{tikzpicture}[remember picture,overlay]
        \node[anchor=north east, xshift=-0.3cm, yshift=-0.05cm, inner sep=0pt] at (current page.north east) {
            \includegraphics[height=0.9cm, keepaspectratio]{../logo2.jpg}
        };
    \end{tikzpicture}%
}

% --- 2. CONFIGURATION FOR FIRST PAGE ---
\newcommand{\FirstPageLayout}{
    \begin{tikzpicture}[remember picture,overlay]
        \node[anchor=north east, xshift=-0.3cm, yshift=-0.05cm, inner sep=0pt] at (current page.north east) {
            \includegraphics[height=1.4cm, keepaspectratio]{../logo.png}
        };
        \node[anchor=south west, xshift=0.5cm, yshift=0.5cm] at (current page.south west) {
            \textcolor{MyGreen}{\footnotesize University of Bologna}
        };
    \end{tikzpicture}
}

% --- 3. CONFIGURATION FOR LAST PAGE ---
\newcommand{\LastPageLayout}{
    \begin{tikzpicture}[remember picture,overlay]
        \node[anchor=north east, xshift=-0.3cm, yshift=-0.05cm, inner sep=0pt] at (current page.north east) {
            \includegraphics[height=0.9cm, keepaspectratio]{../logo.png}
        };
        \node[anchor=south west, xshift=0.5cm, yshift=0.5cm] at (current page.south west) {
            \textcolor{MyGreen}{\footnotesize University of Bologna}
        };
    \end{tikzpicture}
}

% --- Info ---
\title[SVS]{Smart Vehicular System}
\subtitle{Lab 1}
\author[Andrea Bedei, Lorenzo Bacchiani]{Andrea Bedei \\ Lorenzo Bacchiani}
\institute[]{University of Bologna}
\date{}

\begin{document}

% --- Slide 1: Title Page ---
\begin{frame}
    \FirstPageLayout
    \titlepage
\end{frame}

% --- Slide 2: Index ---
\begin{frame}[t]{Index}
  \small
  \tableofcontents[hideallsubsections]
\end{frame}

\section{Introduction}

\begin{frame}{Introduction}
\begin{itemize}
  \item Open source simulator designed for research on driving scenarios
  \item Realistic virtual world to test vehicles with controlled weather, lighting, and traffic
  \item Supports training, testing, and validation while reducing costs and risks compared to on road experiments
\end{itemize}
\end{frame}

\section{Why CARLA}

\begin{frame}{Why choose CARLA}
\begin{itemize}
  \item Safe experimentation with complex scenarios without endangering people or hardware
  \item Open source platform with code and protocols available on GitHub
  \item Open digital assets (urban layouts, buildings, vehicles) available for reuse
  \item Multi sensor setup with cameras, LiDAR, radar, GNSS, and other modalities for perception and planning pipelines
\end{itemize}
\end{frame}

\begin{frame}{Why choose CARLA}
\begin{itemize}
  \item Widely adopted by industry and academia with many published baselines and example projects
  \item Active community with extensive documentation and shared troubleshooting resources
  \item Native ROS 2 interface and ROS bridge for ROS 1 and ROS 2 integration
\end{itemize}
\end{frame}

\section{Requirements}

\begin{frame}{CARLA requirements (summary)}
\rowcolors{2}{black!3}{white}

\setlength{\tabcolsep}{7pt}
\renewcommand{\arraystretch}{1.15}

\begin{tabularx}{\textwidth}{@{}>{\raggedright\arraybackslash}p{3.6cm} X@{}}
\rowcolor{MyGreen}
\textcolor{white}{\textbf{Area}} & \textcolor{white}{\textbf{Requirement}} \\
Supported systems & Windows or Linux (64 bit) \\
GPU & Dedicated GPU with at least 6 GB VRAM, 8 GB recommended \\
Disk space & About 20 GB for the packaged simulator \\
Python & Python 3 with pip 20.3+; pygame and numpy for quick start \\
Network & Two TCP ports available, 2000 and 2001 by default \\
\end{tabularx}
\end{frame}

\begin{frame}{Operational constraints}
\begin{itemize}
  \item Setup can be demanding and benefits from familiarity with Python and simulator configuration
  \item Simulation is an approximation of reality, so traffic and pedestrian behavior can differ from real urban dynamics
  \item Official support focuses on Linux and Windows, with limited support for other platforms and architectures
\end{itemize}
\end{frame}

\section{Installation}

\begin{frame}{Get the simulator binaries}
Follow the official quick start and installation guide, then download the simulator binaries.

\begin{itemize}
  \item GitHub releases: \url{https://github.com/carla-simulator/carla/releases}
  \item Campus file share via SMB using Windows Explorer
\end{itemize}

\medskip
\textbf{SMB access from Windows Explorer}

Type the following path in the Explorer address bar:
\begin{center}
\ttfamily\Large \textbackslash\textbackslash137.204.72.11
\end{center}

Use these credentials when prompted:
\begin{itemize}
  \item Username \texttt{sambauser}
  \item Password \texttt{simcar}
\end{itemize}
\end{frame}

\begin{frame}{Run the simulator}
\begin{itemize}
  \item After extracting the package, start CARLA via the graphical launcher or the server executable
  \item Typical workflow: download CARLA, set up the Python client, then run the example scripts once the server is running
  \item Keep the default networking ports available and ensure the firewall does not block the connection
\end{itemize}
\end{frame}

\section{Python environment}

\begin{frame}[fragile]{Python environment setup}
Use a Python environment manager (conda or venv) to keep dependencies isolated.

\textbf{Create a conda environment}
\begin{lstlisting}[style=code,language=bash]
conda create -n carla-env python=3.7
conda activate carla-env
python --version  # Python 3.7.x
\end{lstlisting}

\textbf{Install dependencies and the CARLA Python package}
\begin{lstlisting}[style=code,language=bash]
pip install -r PythonAPI/examples/requirements.txt
pip install carla
\end{lstlisting}
\end{frame}

\section{Exercises}

\begin{frame}[fragile]{Exercise 1: tutorial.py}
\textbf{Prerequisite: CARLA server must be running.}
\begin{lstlisting}[style=code,language=bash]
cd Code/examples
python tutorial.py
\end{lstlisting}
\begin{itemize}
  \item Connects to the server and spawns a vehicle
  \item Enables autopilot and attaches a depth camera
  \item Saves images in \texttt{\_out/}
\end{itemize}
\end{frame}

\begin{frame}[fragile]{Exercise 2: manual\_control.py}
\textbf{Prerequisite: CARLA server must be running.}
\begin{lstlisting}[style=code,language=bash]
cd Code/examples
python manual_control.py
\end{lstlisting}
\begin{itemize}
  \item Keyboard driving with HUD (demo live)
  \item Main keys: \texttt{WASD}, \texttt{P} (autopilot), \texttt{H} (help)
\end{itemize}
\end{frame}

\begin{frame}[fragile]{Exercise 3: dynamic\_weather.py}
\textbf{Prerequisite: CARLA server must be running.}
\begin{lstlisting}[style=code,language=bash]
cd Code/examples
python dynamic_weather.py
\end{lstlisting}
\begin{itemize}
  \item Changes sun position and weather over time
  \item Clear visual feedback for lighting and rain
\end{itemize}
\end{frame}

\begin{frame}[fragile]{Exercise 4: generate\_traffic.py}
\textbf{Prerequisite: CARLA server must be running.}
\begin{lstlisting}[style=code,language=bash]
cd Code/examples
python generate_traffic.py
\end{lstlisting}
\begin{itemize}
  \item Spawns vehicles and pedestrians automatically
  \item Options: \texttt{--number-of-vehicles} and \texttt{--number-of-walkers}
\end{itemize}
\end{frame}

\section{Python API}

\begin{frame}[fragile]{Python API quick connection}
\begin{lstlisting}[style=code,language=Python]
import carla

# Connect to the server
client = carla.Client('localhost', 2000)
world = client.get_world()
\end{lstlisting}
\end{frame}

\section{References}

\begin{frame}{References}
\small
\begin{enumerate}
  \item A Dosovitskiy, G Ros, F Codevilla, A Lopez, V Koltun. CARLA: An Open Urban Driving Simulator. Proceedings of the First Annual Conference on Robot Learning, 2017.
  \item CARLA documentation, version 0.9.15. \url{https://carla.readthedocs.io/en/0.9.15/}
  \item CARLA GitHub repository. \url{https://github.com/carla-simulator/carla}
  \item ROS website. \url{https://www.ros.org}
\end{enumerate}
\end{frame}

% --- Final page ---
\begin{frame}[c]
    \LastPageLayout
    \centering
    \Huge \textcolor{MyGreen}{Thank you for your attention}

    \vspace{1.2cm}
    \normalsize
    \textbf{Andrea Bedei} \\[0.2cm]
    \textbf{Lorenzo Bacchiani}
\end{frame}

\end{document}
